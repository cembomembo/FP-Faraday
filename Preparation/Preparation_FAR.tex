\documentclass[a4paper]{article}

% --- Page layout and spacing ---
\usepackage[top=2.5cm, left=3cm, right=3cm, bottom=3cm]{geometry}
\usepackage[utf8]{inputenc}      % input encoding
\usepackage[T1]{fontenc}         % font encoding
\usepackage[english]{babel}
\usepackage{setspace}
\setlength{\parindent}{0pt}      % paragraph indentation
\setlength{\parskip}{0.8em}      % space between paragraphs
\setstretch{1.2}                 % line spacing
\usepackage{tocloft}             % section spacing in ToC
\setlength{\cftbeforesecskip}{10pt}
\setlength{\cftbeforesubsecskip}{4pt}
\usepackage{titlesec}            % section title spacing
\titlespacing*{\section}{0pt}{5.0ex plus 1ex minus .2ex}{1.0ex plus .2ex}
\titlespacing*{\subsection}{0pt}{3.0ex plus .5ex minus .2ex}{0.8ex plus .2ex}
\titlespacing*{\subsubsection}{0pt}{2.0ex plus .5ex minus .2ex}{0.8ex plus .2ex}

% --- Math and symbols ---
\usepackage{amsmath, amssymb}    % standard math
\usepackage{empheq}              % boxed equations etc.
\DeclareMathOperator{\artanh}{artanh}
\DeclareMathOperator{\sgn}{sgn}
\usepackage{bm}                  % bold math symbols
\usepackage{cancel}              % strikeout in math
\usepackage{siunitx}             % proper units
\DeclareSIUnit\angstrom{\text{Å}}
\renewcommand{\arraystretch}{0.7}

% --- Graphics and floats ---
\usepackage{graphicx}
\usepackage{float}
\usepackage{wrapfig}
\usepackage[justification=centering]{caption}
\usepackage{subcaption}
\captionsetup[figure]{font=small}

% --- Layout helpers ---
\usepackage{boxedminipage}
\usepackage{enumitem}
\usepackage{afterpage}
\usepackage{changepage}
\usepackage{pdfpages}           % include external PDFs
\usepackage{esvect}             % nice vector arrows
\usepackage{hyperref}           % hyperlinks

% --- Bibliography setup ---
\usepackage{csquotes}
\usepackage[backend=biber,style=numeric,sorting=none]{biblatex}
\addbibresource{references.bib}

% --- Fonts ---
\usepackage{lmodern}            % Computer Modern look across TeX distros

\begin{document}

% --- Title page ---
\begin{titlepage}
  \thispagestyle{empty}
  \begin{center}

    % Title
    \vspace*{1cm}
    {\LARGE \textbf{Faraday-Rotation}}\\[1.2cm]

    % Subtitle
    {\large Preparation Report}\\[2cm]

    % Authors
    \large
    \textbf{Cem Boyaci}\\[-1mm]
    {cemb93@zedat.fu-berlin.de}\\[1cm]

    \textbf{Javier Bellido Roldán}\\[-1mm]
    {bellidoroj98@zedat.fu-berlin.de}\\[1cm]

    \textbf{Leon Goldammer}\\[-1mm]
    {lg4278fu@zedat.fu-berlin.de}\\[6cm]

    % Tutor
    \normalsize
    {Tutor: Ralph Püttner}\\[1.2cm]

    % Footer block
    \textbf{Fortgeschrittenenpraktikum, WS 2025/2026}\\
    Berlin, 26.01.2026\\
    Freie Universität Berlin\\
    Fachbereich Physik

  \end{center}
\end{titlepage}

% --- Table of contents ---
\clearpage
\renewcommand*\contentsname{\huge Contents}
{
  \pagenumbering{gobble}
  \tableofcontents
  \clearpage
}
\pagenumbering{arabic}

% --- Introduction ---
\newpage
\setcounter{page}{1}

\section{Introduction}

The Faraday effect is a magneto-optical phenomenon in which the plane of polarization of linearly polarized light rotates while passing through a transparent medium in an axial magnetic field.
For small rotations, the angle $\theta$ scales with the magnetic flux density $B$ and the path length $L$ in the material, $\theta = VBL$, where $V$ is the material- and wavelength-dependent Verdet constant.
Discovered by Michael Faraday in 1845, the effect is widely regarded as one of the first experimental hints that light and electromagnetism are deeply connected.

What makes Faraday rotation especially useful is that it is non-reciprocal, meaning the sense of rotation does not simply cancel when light travels back through the same element.  
This is why Faraday rotators are used in optical isolators, which act like optical diodes and protect lasers from harmful back reflections~\cite{FARAnleitung}.
Faraday rotation is also used as a sensing principle, for example in fiber-optic current sensors in which the polarization rotation induced by the magnetic field of a current-carrying conductor is used to infer the current magnitude~\cite{mihailovic_fos_faraday_2021}.

In this experiment, we measure the Verdet constant of optical glasses at several wavelengths and relate the observed rotation strength to the wavelength dependence of the refractive index using simple dispersion-based models.

\vspace{1em}
\begin{figure}[H]
\centering
\includegraphics[width=0.85\linewidth]{../resources/figures/faraday_effect.png}
\caption{Schematic illustration of the Faraday effect.
Linearly polarized light propagates through a transparent medium in the presence of a magnetic field applied parallel to the propagation direction.
Inside the material, the plane of polarization rotates continuously due to the different phase velocities of left- and right-circularly polarized components.
The total rotation angle is proportional to the magnetic field strength and the optical path length~\cite{wikimedia_faraday_effect}.}
\label{fig:faraday_effect_schematic}
\end{figure}

% --- Physical Principles ---

\section{Physical Principles}

In the following, we develop the Faraday rotation effect from basic polarization optics and wave propagation in matter to the practical signal extraction method used in the experiment.

\subsection{Polarization States of Light}

We consider a monochromatic plane wave propagating in the $+z$ direction, so that the electric field $\mathbf{E}(z,t)$ oscillates transversely in the $x$-$y$ plane.
Linear polarization means that the tip of the electric field vector oscillates back and forth along a fixed line in the transverse plane, for example along the $x$ axis, $E_x(z,t)=E_0\cos(\omega t-kz)$ and $E_y(z,t)=0$.

Circular polarization means that the tip of the electric field vector rotates at constant magnitude in the transverse plane, which occurs when two orthogonal components of equal amplitude have a phase shift of $\pm 90^\circ$.
A convenient representation is
\begin{align}
\text{RCP:}\quad &E_x^{(r)}(z,t)=E_0\cos(\omega t-kz),\quad E_y^{(r)}(z,t)=E_0\sin(\omega t-kz), \\
\text{LCP:}\quad &E_x^{(l)}(z,t)=E_0\cos(\omega t-kz),\quad E_y^{(l)}(z,t)=-E_0\sin(\omega t-kz),
\end{align}
where the opposite sign in $E_y$ corresponds to the opposite sense of rotation.

A central fact for the Faraday effect is that a linearly polarized wave can be written as the superposition of a left- and a right-circularly polarized wave of equal amplitude.  
Intuitively, adding two counter-rotating circular motions of the same strength cancels the rotational character and leaves a back-and-forth oscillation along a fixed direction, i.e. linear polarization.
This decomposition is particularly useful because Faraday rotation arises when the medium causes the left- and right-circular components to accumulate different phases during propagation, which then leads to a rotation of their linear superposition.

\subsection{Phase Accumulation in a Medium and the Refractive Index}

To understand how the refractive index enters the Faraday effect, it is useful to describe light propagation in terms of wave phase.
We consider a monochromatic plane wave propagating in the $+z$ direction and focus on a single transverse field component, for example $E_x(z,t)$.

In vacuum, a monochromatic plane wave can be written as
\begin{equation}
E_x(z,t) = E_0 \cos(\omega t - k_0 z),
\end{equation}
where $\omega$ is the angular frequency and $k_0 = 2\pi / \lambda_0$ is the vacuum wavenumber.
The spatial term $k_0 z$ determines how quickly the phase of the wave changes as it propagates along the $z$ direction.

When light enters a medium with refractive index $n$, its frequency remains unchanged, but its wavelength is reduced to $\lambda = \lambda_0 / n$.
As a result, the phase changes more rapidly with position and the wavenumber increases to
\begin{equation}
k = \frac{2\pi}{\lambda} = \frac{2\pi}{\lambda_0 / n} = n k_0.
\end{equation}
A larger refractive index therefore means that the phase of the wave ``winds up'' faster per unit length inside the material.

After propagating a distance $L$ through the medium, the accumulated spatial phase advance is
\begin{equation}
\phi = kL = n k_0 L = n \frac{2\pi}{\lambda_0} L.
\end{equation}
This phase directly measures how many oscillation cycles the wave completes over the distance $L$.

This situation is illustrated in Fig.~\ref{fig:phase_accumulation}, which shows how the wavelength is shortened inside a dielectric medium and how this leads to a larger phase accumulation over the same physical distance.

\vspace{1em}
\begin{figure}[H]
\centering
\includegraphics[width=0.95\linewidth]{../resources/figures/phase_accumulation.png}
\caption{Propagation of a plane electromagnetic wave through a medium of refractive index $n$.
Inside the medium the wavelength is reduced to $\lambda = \lambda_0/n$, leading to a faster spatial phase accumulation compared to vacuum.
Adapted from~\cite{demtroeder2013_emwellen}.}
\label{fig:phase_accumulation}
\end{figure}

If two waves propagate through the same length $L$ but experience different refractive indices, they accumulate different phases.  
For two polarization components with refractive indices $n_1$ and $n_2$, the resulting phase difference is
\begin{equation}
\Delta\phi = (n_1 - n_2) k_0 L.
\end{equation}
In the Faraday effect, a magnetic field causes left- and right-circularly polarized light to experience slightly different refractive indices, $n_l \neq n_r$, so that a relative phase shift builds up between the two components.
This relative phase shift has a direct geometric consequence for the polarization state.  
Since a linearly polarized wave can be written as the superposition of left- and right-circularly polarized components of equal amplitude, a relative phase shift between these components rotates the resulting linear polarization direction.

For a phase difference $\Delta\phi$, the rotation angle of the polarization plane is
\begin{equation}
\theta = \frac{\Delta\phi}{2},
\end{equation}
which follows by explicitly adding the two circular fields and using trigonometric identities.

Inserting the expression for $\Delta\phi$ yields
\begin{equation}
\theta = \frac{(n_l - n_r) k_0 L}{2} = \frac{(n_l - n_r) (2\pi / \lambda_0) L}{2} = \frac{\pi}{\lambda_0}(n_l - n_r)L,
\end{equation}
which relates the observed Faraday rotation directly to the refractive index difference of the two circular polarization modes.

\subsection{Microscopic Origin of \texorpdfstring{$n_l \neq n_r$}{} in an Axial Magnetic Field}

The refractive index of a transparent medium originates from the interaction between the electric field of the light wave and the bound electrons in the material.
When an electromagnetic wave propagates through a dielectric, its oscillating electric field drives small oscillatory motions of the electrons around their equilibrium positions.
These driven charges re-radiate electromagnetic waves, and the superposition of the incident and re-radiated fields leads to a phase delay of the transmitted wave.
On a macroscopic level, this phase delay is described by the refractive index $n$, which therefore reflects how strongly the electrons in the medium respond to the applied optical field.

In the absence of a magnetic field, the medium is isotropic with respect to the sense of rotation of the electric field.
Left- and right-circularly polarized light then drive equivalent electronic motions, and both polarization components experience the same refractive index.
As a result, $n_l = n_r$ and no Faraday rotation occurs.

Applying a magnetic field $\mathbf{B}$ along the propagation direction introduces a preferred axial direction and breaks this symmetry.
In the classical picture, the electron motion induced by the optical electric field acquires an additional contribution from the magnetic Lorentz force
\begin{equation}
\mathbf{F}_B = -e\,\mathbf{v} \times \mathbf{B},
\end{equation}
where $\mathbf{v}$ is the electron velocity.
This force bends the electron trajectories in a way that depends on the sense of their rotational motion relative to the magnetic field direction.
As a consequence, electrons driven by left- and right-circularly polarized light respond slightly differently, leading to different polarization strengths and phase delays for the two circular components.
On the macroscopic level, this manifests itself as a small difference between the refractive indices $n_l$ and $n_r$, a phenomenon known as circular birefringence.

The same effect can also be understood from a minimal quantum-mechanical viewpoint.
A helpful way to think about this is that the refractive index of a dielectric is ultimately determined by how easily its bound electrons can be excited by an optical electric field.
In typical optical glasses, the strongest electronic excitations correspond to transitions at much higher frequencies, usually in the ultraviolet.
Visible light is therefore far away from these resonances and is transmitted with little absorption.
Nevertheless, even far from resonance the electrons are still driven weakly by the optical field, and this weak response determines the refractive index observed in the visible range.

Because the electronic response depends on how close the driving frequency is to an underlying resonance, the refractive index is generally not constant but depends on wavelength, $n = n(\lambda)$.
This wavelength dependence is known as dispersion and reflects the gradual change of the electronic phase response with optical frequency.
When a magnetic field is applied, the electronic energy levels associated with these resonances are shifted slightly due to the Zeeman effect.
Since left- and right-circularly polarized light carry opposite angular momentum, they couple to different Zeeman-shifted transitions, which are commonly referred to as $\sigma^+$ and $\sigma^-$ transitions.
As a result, the two circular polarization components effectively probe slightly different parts of the dispersive response curve $n(\lambda)$.
This leads to a small difference between the refractive indices experienced by left- and right-circularly polarized light, again resulting in $n_l \neq n_r$.

Both the classical and the quantum pictures therefore lead to the same conclusion.
An axial magnetic field introduces a handedness into the medium and breaks the equivalence between left- and right-circularly polarized light.
This difference in the optical response is encoded in a small refractive index difference $\Delta n = n_l - n_r$, which ultimately gives rise to the observed rotation of the plane of linear polarization.
 
\subsection{Dispersion, the Verdet Constant, and the Becquerel Relation}

Since the refractive index of a dielectric depends on wavelength, $n = n(\lambda)$, a small magnetic-field-induced shift of the underlying electronic response can be described by stating that the two circular polarization components effectively sample the dispersion curve at slightly different wavelengths.
We therefore write
\begin{equation}
n_l \approx n(\lambda + \Delta\lambda), \qquad n_r \approx n(\lambda - \Delta\lambda),
\end{equation}
where $\Delta\lambda$ is small and proportional to the applied magnetic field $B$.
This parameter does not represent a change of the light wavelength itself, but rather encodes the small Zeeman-induced shift of the electronic resonance that determines the dispersive optical response.

For sufficiently small $\Delta\lambda$, the refractive index can be expanded to first order,
\begin{equation}
n(\lambda \pm \Delta\lambda) \approx n(\lambda) \pm \Delta\lambda \frac{dn}{d\lambda}.
\end{equation}
Taking the difference of the two expressions yields
\begin{equation}
\Delta n = n_l - n_r \approx 2 \Delta\lambda \frac{dn}{d\lambda}.
\end{equation}

We now insert this result into the expression for the Faraday rotation angle derived earlier,
\begin{equation}
\theta = \frac{\pi}{\lambda_0} \Delta n\, L.
\end{equation}
This gives
\begin{equation}
\theta \approx \frac{\pi}{\lambda_0} \left(2 \Delta\lambda \frac{dn}{d\lambda}\right) L.
\end{equation}
Since $\Delta\lambda$ is proportional to the magnetic field $B$, the rotation angle is linear in both $B$ and the optical path length $L$.
This motivates writing the Faraday rotation in the form
\begin{equation}
\theta = V(\lambda)\, B\, L,
\end{equation}
which defines the wavelength-dependent Verdet constant $V(\lambda)$.

A more detailed microscopic treatment of the electron dynamics, in which bound electrons are modeled as driven oscillators in the presence of a magnetic field, leads to an explicit expression for $V(\lambda)$.
In this approach, the Zeeman splitting of circular electron motion results in the so-called Becquerel relation,
\begin{equation}
V(\lambda) = -\frac{1}{2}\frac{e}{m_e^* c}\,\lambda\,\frac{dn}{d\lambda},
\end{equation}
where $e$ is the elementary charge, $c$ the speed of light, and $m_e^*$ an effective electron mass characterizing the optical response of the material.
This relation shows explicitly that the strength of Faraday rotation is governed by the dispersion of the refractive index and is therefore strongly wavelength dependent.

\subsection{Modulation Method and Signal Extraction}

The goal of this part of the experiment is to measure very small Faraday rotation angles $\theta$ with high sensitivity.
Instead of directly determining the rotation angle by mechanically adjusting optical components, the rotation is converted into a measurable change of light intensity behind a polarizing element.
This allows the rotation to be extracted from electrical detector signals with high precision.

After passing through the sample, the light is linearly polarized but rotated by the Faraday angle $\theta$.
The analyzer is a second polarizer placed behind the sample, whose purpose is to convert changes in polarization direction into changes in transmitted intensity.
The transmitted intensity behind the analyzer is described by Malus' law,
\begin{equation}
I = I_0 \cos^2 \phi,
\end{equation}
where $\phi$ is the angle between the polarization direction of the light and the transmission axis of the analyzer.

If the polarization is rotated by a small Faraday angle $\theta$, the effective angle becomes $\phi + \theta$, and the transmitted intensity is
\begin{equation}
I(\theta) = I_0 \cos^2(\phi + \theta).
\end{equation}
The analyzer angle $\phi$ is chosen deliberately to maximize the sensitivity of the transmitted intensity to small changes in $\theta$.

The analyzer is set to $\phi = 45^\circ$, because at this angle the slope of the $\cos^2$ function is maximal and approximately linear in the vicinity of the working point.
This means that small polarization rotations lead to the largest possible relative change in intensity.
This behavior is illustrated schematically in Fig.~\ref{fig:malus_slope}.

\vspace{1em}
\begin{figure}[H]
\centering
\includegraphics[width=0.9\linewidth]{../resources/figures/malus_slope.png}
\caption{Schematic illustration of Malus' law $I \propto \cos^2 \phi$.
At $\phi = 45^\circ$ the slope of the curve is maximal and approximately linear, which makes this angle an optimal working point for detecting small polarization rotations.}
\label{fig:malus_slope}
\end{figure}

For small Faraday rotation angles $\theta \ll 1$, the intensity can be expanded around the working point $\phi = 45^\circ$.
Using a first-order Taylor expansion of $\cos^2(\phi + \theta)$, one obtains
\begin{equation}
I(\theta) \approx \frac{I_0}{2} - I_0 \theta.
\end{equation}
This expression shows that the transmitted intensity consists of a constant offset and a term that depends linearly on the rotation angle.
The linear dependence is essential, because it allows the rotation angle to be determined directly from changes in intensity.

During the measurement, the magnetic field is not applied statically but is modulated sinusoidally by driving the solenoid with an alternating current.
As a result, the Faraday rotation angle oscillates in time,
\begin{equation}
\theta(t) = \theta_B \sin(\Omega t),
\end{equation}
where $\theta_B$ is the rotation amplitude and $\Omega$ the modulation frequency.
The intensity behind the analyzer therefore becomes
\begin{equation}
I(t) \approx I_{DC} + I_{AC} \sin(\Omega t),
\end{equation}
with a constant component
\begin{equation}
I_{DC} = \frac{I_0}{2}
\end{equation}
and an oscillating component of amplitude
\begin{equation}
I_{AC} = I_0 \theta_B.
\end{equation}

Experimentally, the light intensity behind the analyzer is detected by a photodiode, which converts the optical signal into an electrical voltage.
This voltage contains both a constant (time-averaged) part and a small oscillating part.
The constant component $I_{DC}$ is measured using a multimeter in DC mode, while the oscillating component is observed with an oscilloscope.

On the oscilloscope, the oscillation is typically read as a peak-to-peak signal.
This quantity is denoted by $I_{SS}$, where ``SS'' refers to the peak-to-peak swing of the signal.
Since the peak-to-peak value is twice the oscillation amplitude, one has
\begin{equation}
I_{SS} = 2 I_{AC}.
\end{equation}
Using $I_0 = 2 I_{DC}$, the rotation amplitude can therefore be written as
\begin{equation}
\theta_B = \frac{I_{SS}}{4 I_{DC}}.
\end{equation}

In real optical systems, the transmitted intensity does not vanish completely even when the polarizers are crossed.
This can be caused by imperfect polarizers, slight misalignments, or stress-induced changes of the polarization state inside the glass.
To account for this, a modified form of Malus' law is used,
\begin{equation}
I(\phi) = (I_{\parallel} - I_{\perp}) \cos^2 \phi + I_{\perp},
\end{equation}
where $I_{\parallel}$ and $I_{\perp}$ are the intensities measured for parallel and crossed polarizers, respectively.

From these quantities, a contrast factor
\begin{equation}
K = \frac{I_{\parallel} - I_{\perp}}{I_{\parallel} + I_{\perp}}
\end{equation}
is defined, which quantifies how well the polarization modulation is preserved.
Including this correction, the experimentally relevant expression for the Faraday rotation amplitude becomes
\begin{equation}
\theta_B = \frac{1}{K} \frac{I_{SS}}{4 I_{DC}}.
\end{equation}

This modulation-based detection method allows very small Faraday rotations to be extracted reliably from electrical signals, even in the presence of slow drifts and imperfections in the optical setup.

\subsection{Magnetic Field Calibration and Length-Averaged Field}

We explain why the finite solenoid field is not uniform along the optical path and why an average field must be used.  
We define $B_{\mathrm{avg}} = (1/l)\int_0^l B(z)\,dz$ and justify this averaging by the linear dependence of $\theta$ on $B$.  

\subsection{Dispersion Model for \texorpdfstring{$n(\lambda)$}{} and Determination of \texorpdfstring{$\lambda_R$ and $N$}{}}

We introduce the one-resonance dispersion model used in the evaluation and its linearized form.  
We explain the straight-line plot of $1/(n^2-1)$ versus $1/\lambda^2$ and how the slope and intercept yield the model resonance wavelength $\lambda_R$.  
Finally, we show how the slope together with the experimentally obtained $m_e^*$ determines the density of dispersion electrons $N$.  

% --- Experimental Setup ---
\section{Experimental Setup}

xxx

% --- Procedure ---
\section{Procedure}

xxx

% --- References ---
\newpage
\setstretch{1.0}
\printbibliography[heading=bibintoc]

\section*{Author's Note}
AI-based writing and programming tools were used in a supporting role to refine the wording of this report and to assist in formatting Python and LaTeX code.
All scientific analysis, data evaluation, and interpretation were carried out independently by the authors.

\end{document}
