\documentclass[a4paper]{article}

% --- Page layout and spacing ---
\usepackage[top=2.5cm, left=3cm, right=3cm, bottom=3cm]{geometry}
\usepackage[utf8]{inputenc}      % input encoding
\usepackage[T1]{fontenc}         % font encoding
\usepackage[english]{babel}
\usepackage{setspace}
\setlength{\parindent}{0pt}      % paragraph indentation
\setlength{\parskip}{0.8em}      % space between paragraphs
\setstretch{1.2}                 % line spacing
\usepackage{tocloft}             % section spacing in ToC
\setlength{\cftbeforesecskip}{10pt}
\setlength{\cftbeforesubsecskip}{4pt}
\usepackage{titlesec}            % section title spacing
\titlespacing*{\section}{0pt}{5.0ex plus 1ex minus .2ex}{1.0ex plus .2ex}
\titlespacing*{\subsection}{0pt}{3.0ex plus .5ex minus .2ex}{0.8ex plus .2ex}
\titlespacing*{\subsubsection}{0pt}{2.0ex plus .5ex minus .2ex}{0.8ex plus .2ex}

% --- Math and symbols ---
\usepackage{amsmath, amssymb}    % standard math
\usepackage{empheq}              % boxed equations etc.
\DeclareMathOperator{\artanh}{artanh}
\DeclareMathOperator{\sgn}{sgn}
\usepackage{bm}                  % bold math symbols
\usepackage{cancel}              % strikeout in math
\usepackage{siunitx}             % proper units
\DeclareSIUnit\angstrom{\text{Å}}
\renewcommand{\arraystretch}{0.7}

% --- Graphics and floats ---
\usepackage{graphicx}
\usepackage{float}
\usepackage{wrapfig}
\usepackage[justification=centering]{caption}
\usepackage{subcaption}
\captionsetup[figure]{font=small}

% --- Layout helpers ---
\usepackage{boxedminipage}
\usepackage{enumitem}
\usepackage{afterpage}
\usepackage{changepage}
\usepackage{pdfpages}           % include external PDFs
\usepackage{esvect}             % nice vector arrows
\usepackage{hyperref}           % hyperlinks

% --- Bibliography setup ---
\usepackage{csquotes}
\usepackage[backend=biber,style=numeric,sorting=none]{biblatex}
\addbibresource{references.bib}

% --- Fonts ---
\usepackage{lmodern}            % Computer Modern look across TeX distros

\begin{document}

% --- Title page ---
\begin{titlepage}
  \thispagestyle{empty}
  \begin{center}

    % Title
    \vspace*{1cm}
    {\LARGE \textbf{Faraday-Rotation}}\\[1.2cm]

    % Subtitle
    {\large Preparation Report}\\[2cm]

    % Authors
    \large
    \textbf{Cem Boyaci}\\[-1mm]
    {cemb93@zedat.fu-berlin.de}\\[1cm]

    \textbf{Javier Bellido Roldán}\\[-1mm]
    {bellidoroj98@zedat.fu-berlin.de}\\[1cm]

    \textbf{Leon Goldammer}\\[-1mm]
    {lg4278fu@zedat.fu-berlin.de}\\[6cm]

    % Tutor
    \normalsize
    {Tutor: Ralph Püttner}\\[1.2cm]

    % Footer block
    \textbf{Fortgeschrittenenpraktikum, WS 2025/2026}\\
    Berlin, 26.01.2026\\
    Freie Universität Berlin\\
    Fachbereich Physik

  \end{center}
\end{titlepage}

% --- Table of contents ---
\clearpage
\renewcommand*\contentsname{\huge Contents}
{
  \pagenumbering{gobble}
  \tableofcontents
  \clearpage
}
\pagenumbering{arabic}

% --- Introduction ---
\newpage
\setcounter{page}{1}

\section{Introduction}

The Faraday effect is a magneto-optical phenomenon in which the plane of polarization of linearly polarized light rotates while passing through a transparent medium in an axial magnetic field.
For small rotations, the angle $\theta$ scales with the magnetic flux density $B$ and the path length $L$ in the material, $\theta = VBL$, where $V$ is the material- and wavelength-dependent Verdet constant.
Discovered by Michael Faraday in 1845, the effect is widely regarded as one of the first experimental hints that light and electromagnetism are deeply connected.

What makes Faraday rotation especially useful is that it is non-reciprocal, meaning the sense of rotation does not simply cancel when light travels back through the same element.  
This is why Faraday rotators are used in optical isolators, which act like optical diodes and protect lasers from harmful back reflections~\cite{FARAnleitung}.
Faraday rotation is also used as a sensing principle, for example in fiber-optic current sensors in which the polarization rotation induced by the magnetic field of a current-carrying conductor is used to infer the current magnitude~\cite{mihailovic_fos_faraday_2021}.

In this experiment, we measure the Verdet constant of optical glasses at several wavelengths and relate the observed rotation strength to the wavelength dependence of the refractive index using simple dispersion-based models.

% --- Physical Principles ---

\section{Physical Principles}

In the following, we develop the Faraday rotation effect from basic polarization optics and wave propagation in matter to the practical signal extraction method used in the experiment.

\subsection{Polarization States of Light}

We consider a monochromatic plane wave propagating in the $+z$ direction, so that the electric field $\mathbf{E}(z,t)$ oscillates transversely in the $x$-$y$ plane.
Linear polarization means that the tip of the electric field vector oscillates back and forth along a fixed line in the transverse plane, for example along the $x$ axis, $E_x(z,t)=E_0\cos(\omega t-kz)$ and $E_y(z,t)=0$.

Circular polarization means that the tip of the electric field vector rotates at constant magnitude in the transverse plane, which occurs when two orthogonal components of equal amplitude have a phase shift of $\pm 90^\circ$.
A convenient representation is
\begin{align}
\text{RCP:}\quad &E_x^{(r)}(z,t)=E_0\cos(\omega t-kz),\quad E_y^{(r)}(z,t)=E_0\sin(\omega t-kz), \\
\text{LCP:}\quad &E_x^{(l)}(z,t)=E_0\cos(\omega t-kz),\quad E_y^{(l)}(z,t)=-E_0\sin(\omega t-kz),
\end{align}
where the opposite sign in $E_y$ corresponds to the opposite sense of rotation.

A central fact for the Faraday effect is that a linearly polarized wave can be written as the superposition of a left- and a right-circularly polarized wave of equal amplitude.  
Intuitively, adding two counter-rotating circular motions of the same strength cancels the rotational character and leaves a back-and-forth oscillation along a fixed direction, i.e. linear polarization.
This decomposition is particularly useful because Faraday rotation arises when the medium causes the left- and right-circular components to accumulate different phases during propagation, which then leads to a rotation of their linear superposition.

\subsection{Phase Accumulation in a Medium and the Refractive Index}

To understand how the refractive index enters the Faraday effect, it is useful to describe light propagation in terms of wave phase.
We consider a monochromatic plane wave propagating in the $+z$ direction and focus on a single transverse field component, for example $E_x(z,t)$.

In vacuum, a monochromatic plane wave can be written as
\begin{equation}
E_x(z,t) = E_0 \cos(\omega t - k_0 z),
\end{equation}
where $\omega$ is the angular frequency and $k_0 = 2\pi / \lambda_0$ is the vacuum wavenumber.
The spatial term $k_0 z$ determines how quickly the phase of the wave changes as it propagates along the $z$ direction.

When light enters a medium with refractive index $n$, its frequency remains unchanged, but its wavelength is reduced to $\lambda = \lambda_0 / n$.
As a result, the phase changes more rapidly with position and the wavenumber increases to
\begin{equation}
k = \frac{2\pi}{\lambda} = \frac{2\pi}{\lambda_0 / n} = n k_0.
\end{equation}
A larger refractive index therefore means that the phase of the wave ``winds up'' faster per unit length inside the material.

After propagating a distance $L$ through the medium, the accumulated spatial phase advance is
\begin{equation}
\phi = kL = n k_0 L = n \frac{2\pi}{\lambda_0} L.
\end{equation}
This phase directly measures how many oscillation cycles the wave completes over the distance $L$.

This situation is illustrated in Fig.~\ref{fig:phase_accumulation}, which shows how the wavelength is shortened inside a dielectric medium and how this leads to a larger phase accumulation over the same physical distance.

\vspace{1em}
\begin{figure}[H]
\centering
\includegraphics[width=0.75\linewidth]{../resources/figures/phase_accumulation.png}
\caption{Propagation of a plane electromagnetic wave through a medium of refractive index $n$.
Inside the medium the wavelength is reduced to $\lambda = \lambda_0/n$, leading to a faster spatial phase accumulation compared to vacuum.
Adapted from~\cite{demtroeder2013_emwellen}.}
\label{fig:phase_accumulation}
\end{figure}

If two waves propagate through the same length $L$ but experience different refractive indices, they accumulate different phases.
For two polarization components with refractive indices $n_1$ and $n_2$, the resulting phase difference is
\begin{equation}
\Delta\phi = (n_1 - n_2) k_0 L.
\end{equation}
In the Faraday effect, a magnetic field causes left- and right-circularly polarized light to experience slightly different refractive indices, $n_l \neq n_r$, so that a relative phase shift builds up between the two components.

\subsection{Microscopic Origin of $n_l \neq n_r$ in an Axial Magnetic Field}
We present the classical picture in which the magnetic field introduces a handedness through the Lorentz force and leads to different responses for left- and right-circular driving fields.  
We then give a minimal quantum interpretation via Zeeman splitting and the association of $\sigma^{\pm}$ transitions with opposite circular polarizations, motivating why $n_l$ and $n_r$ differ in a magnetic field.  

\subsection{Dispersion, the Verdet Constant, and the Becquerel Relation}
We connect the small difference $\Delta n = n_l-n_r$ to the slope of the dispersion curve using the linear approximation $\Delta n \approx \Delta\lambda\, dn/d\lambda$.  
We derive $\theta = VBL$ and the Becquerel form $V \propto \lambda\, dn/d\lambda$ including a discussion of the sign convention and the physical meaning of the effective oscillator mass $m_e^*$.  

\subsection{Modulation Method and Signal Extraction}
We start from Malus' law and show why the working point $\phi = 45^\circ$ maximizes sensitivity.  
We derive the relation between the rotation amplitude $\theta_B$ and the measured detector signals, clarifying the difference between amplitude and peak-to-peak values and obtaining $\theta_B = I_{SS}/(4I_{DC})$ in the ideal case.  
We then introduce the modified Malus law with leakage intensity, define the contrast factor $K$, and derive the corrected expression $\theta_B = (1/K)\, I_{SS}/(4I_{DC})$ together with the exact procedure to set $\phi=45^\circ$ via $I_{\parallel}$ and $I_{\perp}$.  

\subsection{Magnetic Field Calibration and Length-Averaged Field}
We explain why the finite solenoid field is not uniform along the optical path and why an average field must be used.  
We define $B_{\mathrm{avg}} = (1/l)\int_0^l B(z)\,dz$ and justify this averaging by the linear dependence of $\theta$ on $B$.  

\subsection{Dispersion Model for $n(\lambda)$ and Determination of $\lambda_R$ and $N$}
We introduce the one-resonance dispersion model used in the evaluation and its linearized form.  
We explain the straight-line plot of $1/(n^2-1)$ versus $1/\lambda^2$ and how the slope and intercept yield the model resonance wavelength $\lambda_R$.  
Finally, we show how the slope together with the experimentally obtained $m_e^*$ determines the density of dispersion electrons $N$.  

% --- Experimental Setup ---
\section{Experimental Setup}

xxx

% --- Procedure ---
\section{Procedure}

xxx

% --- References ---
\newpage
\setstretch{1.0}
\printbibliography[heading=bibintoc]

\section*{Author's Note}
AI-based writing and programming tools were used in a supporting role to refine the wording of this report and to assist in formatting Python and LaTeX code.
All scientific analysis, data evaluation, and interpretation were carried out independently by the authors.

\end{document}
